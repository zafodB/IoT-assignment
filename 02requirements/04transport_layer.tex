\subsection{Transport layer}
The transport layer is the fourth layer in \acrshort{osi} model and the last layer covered in this analysis. It provides end-to-end communication services to the upper layer (or application)\cite{Braden1989RequirementsLayers}. To achieve its goal, it uses services provided by lower layers, however, it is not concerned with the specifics of operations of lower layers. Most widely adopted transport layer protocols are \acrshort{udp} and \acrshort{tcp}, while \acrshort{tcp} is part of the TCP/IP protocol suite and can thus be considered a cornerstone of the Internet.

As transport layers protocols' focus is to transport data from one end point to another, ideally as fast as possible, one of the main problems that may occur is congestion. Congestion from the perspective of transport layer occurs, when the source of the data is producing and transmitting data faster than the destination can process it. When the processing buffer at the destination fills up and overflows, resulting in packet loss, retransmission of the lost packets is needed. Some transport layer protocols avoid this by implementing a flow/congestion control mechanism. This mechanism adjusts the rate at which data is transmitted, so that the receiving end has enough time process it.

Another services offered by some transport layer protocols are orderly transmission and loss recovery \cite{Kuzuno2017BlockchainBitcoin}. Both can be in principle solved by application layer, but in some scenarios it may be more suitable to implement these in transport layer directly. Orderly transmission ensures that packets are delivered in correct order. This can be achieved by introducing numbers marking order in which packets were sent -- \textit{sequence numbers} in the transport layer protocol header. Loss recovery detects loss of data due to congestion and ensures that the lost packets are retransmitted. Sequence numbers may also be used for this purpose.

When used in a \acrshort{wsn}, the main concern of transport layer protocols is to achieve reasonable throughput, while maintaining the energy consumption as low as possible. The main source of energy consumption is retransmission due to congestion.

\acrshort{tcp} and \acrshort{udp} are different and cater to different needs. When used in a ordinary device, a simple analysis of their strengths and weaknesses can be sufficient to advocate use of one or another. However, when used in wireless sensor networks, neither seem to be suitable. As Kuzuno and Karam note, there are several disadvantages to both protocols \cite{Kuzuno2017BlockchainBitcoin}. In the folloing section, we will therefore consider transport layer protocols designed specifically for \acrshort{wsn}.