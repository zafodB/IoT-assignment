\section{Introduction}
% 
In today’s world of outsourcing production to other country than the one where the development was done in, is a common practice. Reasons varies, but mostly, it is the economical aspect. This however, raises many problems within logistics, meaning getting products from the plant on one side of the globe to a warehouses and customers on the other. Keeping track of every container leaving the plant on truck’s trailer to a port, to be loaded on a huge cargo ship with hundreds of others and subsequently being unloaded at the destination port to the right truck’s trailer and getting it to a given warehouse, poses a challenge. What is even more, as a wide range of products are being transported, each product requires a different kind of handling in means of temperature in a container, fragility of products or speed of shipment. But how to collect all of these data and provide enhanced supply chain management?
One of ways to go is the use of \acrshort{iot}\footnotemark. ``\textit{The \acrfull{iot} is a system of interrelated computing devices, mechanical and digital machines, objects, animals or people that are provided with unique identifiers and the ability to transfer data over a network without requiring human-to-human or human-to-computer interaction}'' \cite{MargaretRouse2016WhatWhatIs.com}. In the given problem in logistics, each object – container, is able to provide loads of useful data which may be collected. Collection of such data may be done with help of various sensors which may be deployed in each container. By interconnecting them, each container will have its own small network, made of several slave sensors and a master device. In this use case, the focus will be put into the use of such networks on cargo ships to provide a crew with information from each container in a central system, as well as providing a shipment client with details about his container.
% 
\footnotetext{\url{http://www.businessinsider.com/internet-of-things-logistics-supply-chain-management-2016-10?r=US&IR=T&IR=T}, accessed 05-05-2018}

To address this problem, the suggested solution is to equip each container with a hub and a set of sensors which may be added on demand, interconnecting the hub with a central receiver and providing information to a central system. In this project, the main area of research will be analysis of different wireless standards which may fit the solution as well as design of architecture of such a system.

Lets take John as an example. John is the owner of a shop chain based in Denmark, selling party decorations, costumes and gadgets. It is also a country, where their own designs are. However, as it is cheaper for them to have a production in a country with cheap work force, they are produced in China, from where all other stuff are bought as well. John would like to keep track of all containers shipped from China to Denmark, as well as being updated about conditions within container. Therefore, he decides that all of them should be equipped with \acrshort{gps}, temperature monitor, motion sensor and of course a \acrshort{gsm} module with hub. Because of that, he can track where a container is live, check whether the requested temperature for transportation is met, see when a container is being opened and have access to this information anytime and anywhere in the world. This a good solution for when a container is on land. However, when it is loaded on a cargo ship, sailing on the ocean, there is no signal from base station and therefore communication with satellite would be needed which is relatively expensive\footnotemark.
% 
\footnotetext{\url{https://www.quora.com/Why-is-satellite-Internet-service-so-relatively-expensive-and-not-available-worldwide}, accessed 05-05-2018}

One of many solutions to this problem, would be having a cargo ship equipped with several hubs to which containers on board would be periodically pushing information and these will be collected in the central system. This information then, would be every few minutes sent via satellite to central server and accessible for a client. Because of that, the cost for internet connection on board would be lower as there would be only one from the central system instead of many from each container, and crew would have precise information about each container, knowing if there was some problem.

The rest of this report is as follows: in Section \ref{sec:layers} we analyse the four layers of a \acrlong{wsn}